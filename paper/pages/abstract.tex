\chapter{\abstractname}

Generative models and in particular Generative Adversarial Networks (GANs) have become very popular and powerful data generation tool. In recent years, major progress has been made in extending this concept into the quantum realm. In this work, we take a closer look at two different approaches to quantum GANs. We show how they can be used to learn and generate quantum states, that were supplied in the input set and seen at the training time. We also propose a new quantum-classical hybrid method, that overcomes the limitation of the current approaches. It allows to learn the distribution of the supplied states and generate new states, that were not a part of the input set, but come from the learned distribution. We conduct numerous numerical experiments showing, how quantum GANs and our method perform on different types of input states and with different architectures.






%\makeatletter
%\ifthenelse{\pdf@strcmp{\languagename}{english}=0}
%{\renewcommand{\abstractname}{Kurzfassung}}
%{\renewcommand{\abstractname}{Abstract}}
%\makeatother
%
%\chapter{\abstractname}
%
%%TODO: Abstract in other language
%\begin{otherlanguage}{ngerman} % TODO: select other language, either ngerman or english !
%
%\end{otherlanguage}
%
%
%% Undo the name switch
%\makeatletter
%\ifthenelse{\pdf@strcmp{\languagename}{english}=0}
%{\renewcommand{\abstractname}{Abstract}}
%{\renewcommand{\abstractname}{Kurzfassung}}
%\makeatother