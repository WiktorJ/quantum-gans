\chapter{\abstractname}

Generative models and in particular Generative Adversarial Networks (GANs) have
become very popular and powerful data generation tool. In recent years, a major
progress has been made in extending this concept into the quantum realm. In this
work we take a closer look at two different approaches to quantum GANs. We show how they can
be used to learn previously seen pure and mixed quantum state. We also propose a
new quantum-classical hybrid method, that overcomes the limitation of
the previous approaches and allows to generate unseen before quantum state. We
conduct a numerous numerical experiments showing, how quantum 
GANs and our method perform on different types of inputs and with different
architectures. 






%\makeatletter
%\ifthenelse{\pdf@strcmp{\languagename}{english}=0}
%{\renewcommand{\abstractname}{Kurzfassung}}
%{\renewcommand{\abstractname}{Abstract}}
%\makeatother
%
%\chapter{\abstractname}
%
%%TODO: Abstract in other language
%\begin{otherlanguage}{ngerman} % TODO: select other language, either ngerman or english !
%
%\end{otherlanguage}
%
%
%% Undo the name switch
%\makeatletter
%\ifthenelse{\pdf@strcmp{\languagename}{english}=0}
%{\renewcommand{\abstractname}{Abstract}}
%{\renewcommand{\abstractname}{Kurzfassung}}
%\makeatother