\chapter{Unseen Quantum State Generation}\label{chapter:my_contribution}
In the previous chapter we evaluated two different types of quantum GANs. The
common problem of those is the inability to generate new, unseen states. In this
chapter we propose the hybrid classical-quantum framework that can overcome this limitation. 

Our idea is based WQGANs and how the quantum Wasserstein distance is
approximated during the training. The discriminator at every step approximates
the distance between some fixed, real input state and the generated state which
changes after each iteration. However, the discriminator never needs an access to
the actual real input state, it only operates on the set of measured
expectations.

Given a parametrized circuits $U$
and set of parameters $\Theta = \{\theta_i\}$ and a set of operators $H =
\{H_j\}$, we prepare a set of vectors of expectations $S$. Each vector $s_{\theta_i}
\in S$ contains the expectations of the circuit $U(\theta_i)$, such that
$s_{\theta_i}^{(j)} = \langle H_j \rangle_{U(\theta_i)} $.

Assuming that all the vectors in $S$ come from the same distribution $p_S$,
the proposed framework is defined in two parts as follows:
\begin{enumerate}
\item \textit{Classical}: Takes as the input the set $S$ and uses it to learn a function $f:
  \mathbb{R}^{n} \to \mathbb{R}^{|H|}$. Given an arbitrary vector $z \in
  \mathbb{R}^n$ (e.g. random noise), this function produces a new vector $s' =
  f(z)$ such that $s' \sim p_S$.  
\item \textit{Quantum}: Takes $s'$ as the input and uses it as
  expectations of real input source in the WQGANs setting described in the
  previous chapter. The generator trained using $s'$ produces new, unseen before
  quantum state.
\end{enumerate}

Once the function $f$ is learnt, it can be used arbitrary many times to produce
new vectors of expectations.
With those vectors, it is possible to generate new quantum states that come from
some circuit $U(\theta')$, without ever knowing $U$ or $\theta'$.
In the following sections, we show how exactly the function $f$ can be obtained
for two different cases. 
\section{Labeled State Generation}
If the quantum state produced by circuit $U$ can be labeled by some continuous
variable, we can use this fact to find the function $f$. Specifically, here we
assume that there exist $g \in \mathbb{R}$, such that each element of parameters
vector $\theta_i$ is a function of $g$, i.e. $\theta_i = [\theta^{(1)}(g),
\theta^{(2)}(g), \ldots, \theta^{(k)}(g)]$. 

\section{Unlabeled State Generation}

%%% Local Variables:
%%% mode: latex
%%% TeX-master: "../main"
%%% End: