\chapter{Conclusions}\label{chapter:conclusions}
The field of quantum machine learning is currently in an early, exploratory
stage. There have been many attempts to bring the successful classical machine
learning ideas into the quantum realm. In this work we took a closer look at
Generative Adversarial Networks and their realization on the quantum machines.
We evaluated two types of quantum GANs, SQGANs \cite{Dallaire_Demers_2018} and
WQGANs \cite{depalma2020quantum}, using pure and mixed target states of
different sizes and circuit designs. Both of those networks are able to learn to
generate previously seen quantum states for small target circuits.
However, according to our evaluation, WQGANs are more robust, easier to train
and work for wider inputs. We also confirmed, that the quantum
Wasserstein distance approximated by WQGANs is closely correlated with the fidelity
and is a good metric to measure the similarity of quantum states.

Finally, by using the insights from the above evaluation, we proposed the new way
to generate unseen quantum states. We combined the classical generative modeling 
with WQGANs to train parametrized quantum circuit able to generate the unseen quantum
states. We showed experimentally, that the quantum states generated by our
method follow the distribution of the states used to train the generative networks.

All the attempts so far, including this work, concentrated on small input of
several or maximum dozen qubits. An important are to explore is the
scalability of quantum GANs for wider inputs, especially on the currently available NISQ machines.  
Another interesting question is, how could the unseen states be generated
in the fully quantum manner, without the need to use the classical computer. 

%%% Local Variables:
%%% mode: latex
%%% TeX-master: "../main"
%%% End: