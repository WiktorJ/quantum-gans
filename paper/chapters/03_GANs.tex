\chapter{Generative Adversarial Networks Introduction}\label{chapter:gans}
Generative Adversarial Networks (GANs)\cite{goodfellow2014generative} is a
machine learning framework designed to estimate generative models using
adversarial process. At the core it consists of two models: the generative one
$G$ which is capable of learning the distribution of the provided data and
discriminative model $D$ that, given a data point, estimates whether it comes
from input data or was generated by $G$. The models are set to compete with each
other in a minmax game. $D$ is trained to maximize the probability of correctly
distinguishing between generated and real samples, while $G$ is trained to
minimize it.  
\section{Standard GANs}
\section{Waserstein GANs (WGANs)}

%%% Local Variables:
%%% mode: latex
%%% TeX-master: "../main"
%%% End: