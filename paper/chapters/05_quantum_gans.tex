\chapter{Quantum Generative Adversarial Networks}\label{chapter:quantum_gans}
The field of Quantum Machine Learning (QML) is still in very early stages and there
has been an ongoing effort on translating the  classical Machine Learning (ML)
concepts into QML realm. Because of the limitations of current quantum computers
(NISQ) \cite{bharti2021noisy} and overall different paradigm of Quantum
Computing (QC), this process is difficult and there is no clear
answer to the question ``how this concept should be realized on QC device?''.
While in classical GANs generator and discriminator are realized as deep neural
networks, NISQ devices are not yet powerful enough to support such architecture.
Instead parametric quantum circuits \cite{Schuld_2020} are used.

In this chapter we take a closer look at two different designs of quantum GANs.
We briefly explain theory behind them and show the results of our evaluation.
The quantum GANs introduced in this chapter are the base of our hybrid
classical-quantum generative framework.

\section{Standard Quantum GANs (SQGANs)}
In the classical GANs, both the discriminator and generator are deep, general
purpose neural networks. The logical extension of this design in the quantum
realm is to model those are general purpose parametric quantum circuits. This
idea was formalized by Dallaire-Demers et al. \cite{Dallaire_Demers_2018} in
the architecture we call SQGANs (Figure \ref{fig:SQGANs_circuit}). The generator
starts in the state $\ket{0}^{\otimes n}$ in the \textbf{Our R/G} wire, where
$n$ is the dimension of the real quantum samples.
Additionally, the generator takes the label state $\ket{\lambda}$ in the
\textbf{Label R/G} wire and the random state $\ket{z}$ that provides the entropy in
the \textbf{Batch R/G} wire. The label state $\ket{\lambda}$ lets the
generator to learn the conditional distribution $p_g(x|\lambda)$ instead of $p_g(x)$,
this design was inspired by classical Conditional GANs
\cite{mirza2014conditional}. The discriminator takes the generated state
$\rho_\lambda^G$ or real state $\rho_\lambda^R$ and the corresponding label
$\ket{\lambda}$ in the \textbf{Label D} wire, it also uses the workspace
\textbf{Batch D} initialized in $\ket{0}$ state. The measurement on the single
qubit wire \textbf{Out D} corresponds to the probability of the state in 
\textbf{Out R/G} being real or generated.

There are not any particular restrains on how $G$ and $D$ circuits should look
like. However, to be able to learn and differentiate between arbitrary quantum
states the ansatz used should be universal (i.e. be able to generate every
quantum state given appropriate depth). The ansatz used in this
work is described in details in Appendix \ref{apx:sqgans_ansatz}.

\begin{figure}[htbp!]
  \begin{tikzcd}
    \rstick{Out D} &&&&&& \rstick{$\ket{0}$} &&& \qw & \qw & \gate[4, disable auto height]{D(\theta_D)} & \meter{} \\
    \rstick{Batch D} &&&&&& \rstick{$\ket{0}^{\otimes d}$} &&& \qw & \qw & & \qw \\
    \rstick{Label D} &&&&&& \rstick{$\ket{\lambda}$} &&& \qw & \qw & & \qw \\ 
    \rstick{Out R/G} &&&&&& \rstick{$\ket{0}^{\otimes n}$} &&& \gate[3, disable auto height]{
      \begin{array}{c}
      R \\ or \\ G(\theta_G)
      \end{array}
    } & \qw \rho_{\lambda}^{R/G} & & \qw \\ 
    \rstick{Label R/G} &&&&&& \rstick{$\ket{\lambda}$} &&& \qw & \qw & \qw & \qw \\ 
    \rstick{Batch R/G} &&&&&& \rstick{$\ket{z}$} &&& \qw & \qw & \qw & \qw 
  \end{tikzcd}
  \caption{SQGANs schema. The discriminator $D$ and the generator $G$ are
    parametric quantum circuits. The first 3 wires go
    directly to the discriminator. \textbf{Out D} outputs the probability of the
  input being generated. \textbf{Batch D} is an additional workspace of the
  discriminator and \textbf{Label D} contains the label state. \textbf{Out R/G}
  carries the generated or real state. \textbf{Label R/G} contains the label
  state and \textbf{Batch R/G} is the noise source for the generator, not used
  with the real samples.\label{fig:SQGANs_circuit} }
\end{figure}

\subsection{Training}
As in the SGANs we are interested in the minmax game setting. Specifically, in
the \textbf{Out D} wire, the measurement of $\ket{1}$ indicates that the sample
was real and $\ket{0}$ that the state was generated by $G$.
\footnote{Using $\ket{1}$ and $\ket{0}$ in this order is just a convention we
  assume. Any other orthogonal pair can be used.}
This should be the case for each label state $\ket{\lambda}$, which gives the
objective \ref{eq:SQGANs_objective}. 
\begin{equation}
  \begin{split}
  \label{eq:SQGANs_objective}
  & \max_{D}\min_{G} \mathcal{L}(G, D) = \\
  & \max_{D}\min_{G}  \frac{1}{\Lambda}\sum_{\lambda \in \Lambda}{P((D(\theta_D, \ket{\lambda}, R(\lambda)) = \ket{1}) \land (D(\theta_D, \ket{\lambda}, G(\theta_G, \ket{\lambda}, \ket{z}) = \ket{0})))}
  \end{split}
\end{equation} 
Where the discriminator objective is to maximize the probability of measuring $\ket{1}$
given real sample and measuring $\ket{0}$ given generated state. At the same
time for the generator the objective is to do the opposite.

The SQGANs cost function \ref{eq:SQGANs_objective} in contrast to the SGANs cost
function \ref{eq:SGANs_objective} is not defined with log-likelihood. In quantum
setup is more natural to work with linear functions and since the $\log$ is
convex, the optimum is the same for both.

The cost function $\mathcal{L}(G, D)$ expressed in terms of measurements and
assuming equal probability of sampling from $R$ and $G$ takes the form \ref{eq:SQGANs_objective}

\begin{equation}
  \label{eq:SQGAN_objective_trace}
  \mathcal{L}(G, D) = \frac{1}{2} + \frac{1}{4\Lambda}\sum_{\lambda \in
  \Lambda}{}tr((\rho_\lambda^{DR}(\theta_D) - \rho_\lambda^{DG}(\theta_D, \theta_G, z))Z)
\end{equation}
For detailed derivation on how to get from \ref{eq:SQGANs_objective} to
\ref{eq:SQGANs_objective} refer to Appendix \ref{apx:sqgans_cost_function}.

\subsubsection{Gradient Estimation}
To optimize the parameters $\theta_D$ and $\theta_G$ classical gradient descent
method was used. First, the value of the cost function $\mathcal{L}(G, D)$ was
estimated by sampling from the circuit \ref{fig:SQGANs_circuit}. This allows to
calculate the gradient w.r.t. $\theta_D$ and $\theta_G$ on classical computer
and update the parameters at the step $k$ in the following way
\begin{equation}
  \begin{split}
    \theta^{k+1}_D = \theta^{k}_D + \alpha^k_D\nabla_{\theta_D}\mathcal{L}(\theta^k_G, \theta^k_D) \\
    \theta^{k+1}_G = \theta^{k}_G - \alpha^k_G\nabla_{\theta_G}\mathcal{L}(\theta^k_G, \theta^k_D)
  \end{split}
\end{equation}
where $\alpha^k_D$ and $\alpha^k_G$ are learning rate metaparameters that can
depend on the step $k$.

It is also possible to estimate the gradient directly on quantum computer by
creating an explicit quantum circuit for each element of vectors $\theta_D / \theta_G$
and reading the grading by sampling for those circuits \cite{Dallaire_Demers_2018}.
\subsection{Evaluation Results}
\subsubsection{Experimental Setup}
In all of the experiments the real samples are generated by evaluating the
circuit from Figure \ref{fig:phase_circuit_small}. This circuit was constructed
by Smith et al. \cite{smith2020crossing} to study topological phase transitions.
All the gates in the circuit are parameterized by a single real valued parameter
$g \in [-1;1]$. The detailed gates layout is described in Appendix \ref{apx:topological_phase_transition_ansatz}.
\begin{figure}[htbp!]
  \begin{tikzcd}
    \lstick{$\ket{0}$} & \gate[2, disable auto height]{U_1(g)} & \qw & \qw & \qw &
    \qw & \qw & \qw \\
    \lstick{$\ket{0}$} & & \gate[2, disable auto height]{U(g)}  & \qw & \qw & \qw & \qw & \qw \\
    \lstick{$\ket{0}$} & \qw & \qw & \gate[2, disable auto height]{U(g)}  & \qw & \qw & \qw & \qw \\
    \lstick{$\ket{0}$} & \qw & \qw & \qw & \qw & \ldots  \\
    \vdots & & & & &\ldots & \gate[2, disable auto height]{U(g)} & \qw \\
    \lstick{$\ket{0}$} & \qw & \qw & \qw & \qw & \qw & \qw & \qw \\
  \end{tikzcd}
  \caption{The circuit used for generating real samples. All the gates are
    parametrized by a real valued parameter $g \in [-1; 1]$. For detailed
    description of the circuit see Appendix
    \ref{apx:topological_phase_transition_ansatz} \label{fig:phase_circuit_small}}
\end{figure}

The generator and discriminator are both build using the generic circuit
architecture from Appendix \ref{apx:sqgans_ansatz}. The number of layers differs
and is specified for each experiment separately.

In all the experiment we use Adam optimizer \cite{kingma2017adam} with
parameters $\beta_1 = 0.9$, $\beta_2=0.98$, $\hat{\epsilon} = 1e-9$. The
learning rate is calculated as
\begin{equation}
lr = \max{(\exp{(-\frac{(k+200) * \ln{100}}{4000}), 0.01)}
\end{equation}
where $k$ is the optimization step number. The
learning rate to decrease from $\sim 0.8$ to $0.01$ in the first $3800$ steps and then
remains at $0.01$ for the rest of the training. The exact values were derived
experimentally and show overall good convergence. The exact number of epochs and
iterations is specified for each experiment separately.
\subsubsection{Results For Pure State Real Input}
\subsubsection{Results For Mixed State Real Input}
\subsection{Conclusions}


\section{Wasserstein Quantum GANs (WQGANs)}
Theoretical description and experimental results
  
%%% Local Variables:
%%% mode: latex
%%% TeX-master: "../main"
%%% End: