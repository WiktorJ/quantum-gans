\chapter{Quantum Generative Adversarial Networks}\label{chapter:quantum_gans}
The field of Quantum Machine Learning (QML) is still in very early stages and there
has been an ongoing effort on translating the  classical Machine Learning (ML)
concepts into QML realm. Because of the limitations of current quantum computers
(NISQ) \cite{bharti2021noisy} and overall different paradigm of Quantum
Computing (QC), this process is difficult and there is no clear
answer to the question ``how this concept should be realized on QC device?''.
While in classical GANs generator and discriminator are realized as deep neural
networks, NISQ devices are not yet powerful enough to support such architecture.
Instead parametric quantum circuits \cite{Schuld_2020} are used.

In this chapter we take a closer look at two different designs of quantum GANs.
We briefly explain theory behind them and show the results of our evaluation.
The quantum GANs introduced in this chapter are the base of our hybrid
classical-quantum generative framework.

\section{Standard Quantum GANs (SQGANs)}
Theoretical description and experimental results
\section{Wasserstein Quantum GANs (WQGANs)}
Theoretical description and experimental results

%%% Local Variables:
%%% mode: latex
%%% TeX-master: "../main"
%%% End: