\chapter{Quantum Generative Adversarial Networks}\label{chapter:quantum_gans}
The field of Quantum Machine Learning (QML) is still in very early stages and there
has been an ongoing effort on translating the  classical Machine Learning (ML)
concepts into QML realm. Because of the limitations of current quantum computers
(NISQ) \cite{bharti2021noisy} and overall different paradigm of Quantum
Computing (QC), this process is difficult and there is no clear
answer to the question ``how this concept should be realized on QC device?''.
While in classical GANs generator and discriminator are realized as deep neural
networks, NISQ devices are not yet powerful enough to support such architecture.
Instead parametric quantum circuits \cite{Schuld_2020} are used.

In this chapter we take a closer look at two different designs of quantum GANs.
We briefly explain theory behind them and show the results of our evaluation.
The quantum GANs introduced in this chapter are the base of our hybrid
classical-quantum generative framework.

\section{Standard Quantum GANs (SQGANs)}
In the classical GANs, both the discriminator and generator are deep, general
purpose neural networks. The logical extension of this design in the quantum
realm is to model those are general purpose parametric quantum circuits. This
idea was formalized by Dallaire-Demers et al. \cite{Dallaire_Demers_2018} in
the architecture we call SQGANs (Figure \ref{fig:SQGANs_circuit}). The generator
starts in the state $\ket{0}^{\otimes n}$ in the \textbf{Our R/G} wire, where
$n$ is the dimension of the real quantum samples.
Additionally, the generator takes the label state $\ket{\lambda}$ in the
\textbf{Label R/G} wire and the random state $\ket{z}$ that provides the entropy in
the \textbf{Batch R/G} wire. The label state $\ket{\lambda}$ lets the
generator to learn the conditional distribution $p_g(x|\lambda)$ instead of $p_g(x)$,
this design was inspired by classical Conditional GANs
\cite{mirza2014conditional}. The discriminator takes the generated state
$\rho_\lambda^G$ or real state $\rho_\lambda^R$ and the corresponding label
$\ket{\lambda}$ in the \textbf{Label D} wire, it also uses the workspace
\textbf{Batch D} initialized in $\ket{0}$ state. The measurement on the single
qubit wire \textbf{Out D} corresponds to the probability of the state in 
\textbf{Out R/G} being real or generated.

There are not any particular restrains on how $G$ and $D$ circuits should look
like. However, to be able to learn and differentiate between arbitrary quantum
states the ansatz used should be universal (i.e. be able to generate every
quantum state given appropriate depth). To learn more about ansatz used in this
work, refer to Appendix \ref{apx:sqgans_ansatz}.

\begin{figure}[htbp!]
  \begin{tikzcd}
    \rstick{Out D} &&&&&& \rstick{$\ket{0}$} &&& \qw & \qw & \gate[4, disable auto height]{D(\theta_D)} & \meter{} \\
    \rstick{Batch D} &&&&&& \rstick{$\ket{0}^{\otimes d}$} &&& \qw & \qw & & \qw \\
    \rstick{Label D} &&&&&& \rstick{$\ket{\lambda}$} &&& \qw & \qw & & \qw \\ 
    \rstick{Out R/G} &&&&&& \rstick{$\ket{0}^{\otimes n}$} &&& \gate[3, disable auto height]{
      \begin{array}{c}
      R \\ or \\ G(\theta_G)
      \end{array}
    } & \qw \rho_{\lambda}^{R/G} & & \qw \\ 
    \rstick{Label R/G} &&&&&& \rstick{$\ket{\lambda}$} &&& \qw & \qw & \qw & \qw \\ 
    \rstick{Batch R/G} &&&&&& \rstick{$\ket{z}$} &&& \qw & \qw & \qw & \qw 
  \end{tikzcd}
  \caption{SQGANs schema. The discriminator $D$ and the generator $G$ are
    parametric quantum circuits. The first 3 wires go
    directly to the discriminator. \textbf{Out D} outputs the probability of the
  input being generated. \textbf{Batch D} is an additional workspace of the
  discriminator and \textbf{Label D} contains the label state. \textbf{Out R/G}
  carries the generated or real state. \textbf{Label R/G} contains the label
  state and \textbf{Batch R/G} is the noise source for the generator, not used
  with the real samples.\label{fig:SQGANs_circuit} }
\end{figure}

\subsection{Cost Function}
\subsection{Training}
\subsection{Evaluation Results}
\subsection{Conclusions}


\section{Wasserstein Quantum GANs (WQGANs)}
Theoretical description and experimental results
  
%%% Local Variables:
%%% mode: latex
%%% TeX-master: "../main"
%%% End: