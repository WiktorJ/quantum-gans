\chapter{Quantum Computing Notation}\label{chapter:quantum_mechanic_introduction}
%\enlargethispage{\baselineskip}
We use the standard bra-ket notation to represent vectors in complex
$N$-dimensional Hilbert space $\mathbb{C}^N$, i.e.
$\ket{\cdot} \in \mathbb{C}^N$ for column vector and $\bra{\cdot} \in
(\mathbb{C}^{N})^{\dagger}$ for row vector and complex conjugate of
$\ket{\cdot}$.

In this work we operate on quantum states composed of qubits. A single qubit is
represented as a vector $\ket{\psi} \in \mathbb{C}^2$, such that $\ket{\psi} =
\alpha\ket{0} + \beta\ket{1}$, where 
\begin{align*}
  \ket{0} = \begin{pmatrix}
    1 \\
    0 
  \end{pmatrix},\ 
  \ket{1} = \begin{pmatrix}
    0 \\
    1 
  \end{pmatrix},\ 
  |\alpha|^2 + |\beta|^2 = 1,\
  \alpha, \beta \in \mathbb{R}
\end{align*}
We denote a pure quantum state as a tensor product of $n$ qubits, i.e.
$\ket{\psi_1} \otimes \ldots \otimes \ket{\psi_n} = \ket{\psi_1 \ldots \psi_n} =
\ket{\psi} \in
\mathbb{C}^{2^n}$. 
If some system is a mixture of several pure quantum states, we use density
operator to describe it. The density operator $\rho$ is defined as $\rho =
\sum_{i=1}^{k}p_i\ket{\psi_i}\bra{\psi_i}$, where $p_i$ is the probability of
system being in state $\ket{\psi_i}$ and $\sum_{i=1}^kp_i = 1$.

An observable is an Hermitian matrix $H \in \mathbb{C}^{2^n \times 2^n}$. We
call $Tr[\rho H]$ the expected value of observable $H$ on state $\rho$, where
$Tr$ is matrix trace operation. Since the quantum state $\rho$ cannot be directly
observed, in practice quantum measurements \cite{10.5555/1972505} are used to
estimate the $Tr[\rho H]$.

An operator, or a quantum gate, is an unitary matrix $U \in \mathbb{C}^{2^n
  \times 2^n}$. The operators are used to describe a transition from one
quantum state to another, i.e. $\ket{\psi'} = U\ket{\psi}$.
Similarly to the quantum states, operators may be stated as a tensor
product of $2 \times 2$ Hermitian matrices, i.e. $H = H_1^{(1)} \otimes \ldots \otimes
H_n^{(n)} \in \mathbb{C}^{2^n \times 2^n}$. For the simplicity of notation we skip the
identity matrices in the tensor product notation, e.g. $H_1^{(1)} \otimes I_2 \otimes H_3^{(2)}
\otimes I_4 \otimes I_5 \otimes H_6^{(3)} = H_1^{(1)} \otimes H_3^{(2)} \otimes
H_6^{(3)}$. Where by convention the subscript indicates on which qubit the
operator acts and the superscript indicates different $2\times 2$ matrix.

In context of this work, particularly important are 3 Pauli Matrices:
\begin{align*}
  X = \begin{pmatrix}
    0 & 1 \\
    1 & 0 
  \end{pmatrix},\ 
  Y = \begin{pmatrix}
    0 & -i \\
    i & 0 
  \end{pmatrix},\ 
  Z = \begin{pmatrix}
    1 & 0 \\
    0 & -1 
  \end{pmatrix}.
\end{align*}

Those matrices are Hermitian and unitary, which means they can be used as
observables and operators. We refer to tensor product of several Pauli Matrices
as Pauli String. We also use the Pauli Matrices to define basic parametrized gates
$R_{\sigma}(\theta) = e^{-i\theta \sigma / 2}$, where $\sigma \in \{X, Y, Z\}$
and $\theta \in \mathbb{R}$ is the parameter.
More complex parametrized gates can be build with linear combinations or tensor
products of Pauli Matrices. 

To describe the evolution of quantum state, we use quantum
circuit representation. The operation $\ket{\psi'} = U_k\ldots U_1\ket{\psi}$ is equivalent
to the circuit \ref{fig:small_circuit}.
\begin{figure}[htbp!]
  \centering
  \begin{tikzcd}
    \lstick{$\ket{\psi}$} & \gate{U_1} & \qw & \ldots && \gate{U_k} & \qw && \lstick{\ket{\psi'}}
  \end{tikzcd}
  \caption{\label{fig:small_circuit}}
\end{figure}
If any of the gates $U$ is parametrized, we call this such circuit a
parametrized quantum circuit.

As a measure of similarity of quantum state we use quantum fidelity defined as
\begin{equation*}
  F(\rho, \sigma) = \left(Tr[\sqrt{\sqrt{\rho}\sigma\sqrt{\rho}}] \right)^2.
\end{equation*}
While not obvious from the equation, this measure is symmetric, i.e. $F(\rho,
\sigma) = F(\sigma, \rho)$.

For further details about the notation and quantum
computing in general, please refer to the book \cite{10.5555/1972505}.


%%% Local Variables:
%%% mode: latex
%%% TeX-master: "../main"
%%% End: