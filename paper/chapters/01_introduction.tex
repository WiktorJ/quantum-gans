\chapter{Introduction} \label{chapter:introduction}
\section{Problem Statement}
Generative Modeling aims to learn a conditional probability $P(X|Z = z)$, where $X$ is some observable variable and $Z$ is a target variable. With knowledge of this conditional probability, it is possible to generate new observations $\bar{x} \in X$. In general case, one would not try to obtain the probability $P(X|Z)$ exactly, but learn an approximation. To do so a set of samples $x \in X$ is necessary to train a generator function $G: Z \to X$ which given a target variable $z \in Z$ generates new observation $\bar{x} \in X$. 

In the generative framework, the variable $X$ is a multidimensional vector, in particular it can be used to describe an arbitrary quantum state. With this setup, given a finite set of quantum states $\mathcal{Q} = \set{\ket{\psi_i}}, \ket{\psi_i} \in X\; \forall_i$ the generator function $G$ prepares a new quantum state $\ket{\hat{\psi}}$. This new quantum state is expected to come from the same distribution as the samples in the input set $\mathcal{Q}$.

The only missing piece in the above description is the target variable $Z$. In
the context of the function $G$ generating the quantum states, we can think
about $Z$ as a label of the generated state. That is, for a specific $z \in Z$
the function $G$ always generates the same $\ket{\hat{\psi}}$.

In this work we evaluate different approaches to find the probability $P(X|Z =
z)$ by learning the function $G$. We also address the limitations of the
existing methods propose a new one that combines quantum and classical
generative modeling.
\section{Previous Work}
There exist many different types of generative models. In this work we focus on
one particular type, namely Generative Adversarial Networks (GANs). First
version of GANs was proposed bu Goodfellow et al.
\cite{goodfellow2014generative} (to which we refer as Standard GANs - SGANs),
since then many different variations of GANs were invented
\cite{mirza2014conditional}\cite{karras2019stylebased}\cite{radford2016unsupervised}.
In context of this work, particularly interesting are Wasserstein GANs
(WGANs)\cite{arjovsky2017wasserstein} which minimize \textit{Earth-Mover} distance
between two probability distribution instead of
\textit{Jensen–Shannon} divergence (see Chapter \ref{chapter:gans}) as in SGANs.

In recent years there has been an increasing interest in realizing Generative
Adversarial Networks in Quantum Computing (QC) realm. Dallaire-Demers et al.
proposed QuGANs \cite{Dallaire_Demers_2018} - Quantum Generative Adversarial Networks
where generator and discriminator are parametrized quantum circuits. Similarly
Benedetti et al. proposed fully quantum GANs for pure state approximation \cite{Benedetti_2019}, but
with different (more suitable for NISQ \cite{Preskill_2018}) learning method.
Hybrid methods were also explored, Zoufal et al. build qGAN \cite{Zoufal_2019} -
with parametrized quantum circuit as the generator and classical neural network
as the discriminator. 

De Palma et al. proposed quantum equivalent of Wasserstein distance of order 1
\cite{depalma2020quantum} which made the Quantum Wasserstein GANs
(QWGANs) \cite{kiani2021quantum} possible. This variation of quantum GANs
consist of the parametrized quantum circuit as the generator and the classical linear
program as the discriminator.

\section{Our Contribution}
There has been a substantial effort in the direction of bringing GANs into the quantum
realm. Nevertheless, this is still very early stage and many more routs are yet
to be explored. In this work we focus on building quantum GANs that can generate
new, unseen before, quantum states. Majority of models proposed so far are only
able to generate the states the has been a part of the training data. Only some
architectures \cite{Dallaire_Demers_2018} account for random noise in the input
that allows to generate unseen states. However, as we discuss later, those are
mostly theoretical and do not seem to work well in practice.


We propose a new quantum-classical hybrid approach that allows to generate
an unlimited number of unseen quantum states. We utilize the fully quantum and fully
classical generative models that work together in one framework. 

We proceed as follows. In Chapter \ref{chapter:quantum_mechanic_introduction} we
establish the quantum computing notation used in the reminder of this work. In
Chapter \ref{chapter:gans} we give a general introduction to classical GANs. In
Chapter \ref{chapter:quantum_gans} we combine the knowledge from the previous
chapters to introduce the concept of quantum GANs and talk more about the
different variations of quantum GANs and their limitations. We also evaluate the
performance of quantum GANs on different input types. In Chapter
\ref{chapter:my_contribution} we introduce and describe in depth our
concept of hybrid quantum-classical generative framework and empirically show the
quality of the unseen states generated by the proposed framework. Finally, in Chapter
\ref{chapter:conclusions} we conclude our finding and talk briefly about the
possible future directions. 

%%% Local Variables:
%%% mode: latex
%%% TeX-master: "../main"
%%% End:
